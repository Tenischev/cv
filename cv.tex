\documentclass[letterpaper,11pt]{article}

\usepackage{latexsym}
\usepackage[empty]{fullpage}
\usepackage{titlesec}
\usepackage{marvosym}
\usepackage[usenames,dvipsnames]{color}
\usepackage{verbatim}
\usepackage{enumitem}
\usepackage[hidelinks]{hyperref}
\usepackage{fancyhdr}
\usepackage[english]{babel}
\usepackage{tabularx}
\input{glyphtounicode}


%----------FONT OPTIONS----------
% sans-serif
% \usepackage[sfdefault]{FiraSans}
% \usepackage[sfdefault]{roboto}
% \usepackage[sfdefault]{noto-sans}
% \usepackage[default]{sourcesanspro}

% serif
% \usepackage{CormorantGaramond}
% \usepackage{charter}


\pagestyle{fancy}
\fancyhf{} % clear all header and footer fields
\fancyfoot{}
\renewcommand{\headrulewidth}{0pt}
\renewcommand{\footrulewidth}{0pt}

% Adjust margins
\addtolength{\oddsidemargin}{-0.5in}
\addtolength{\evensidemargin}{-0.5in}
\addtolength{\textwidth}{1in}
\addtolength{\topmargin}{-.5in}
\addtolength{\textheight}{1.0in}

\urlstyle{same}

\raggedbottom
\raggedright
\setlength{\tabcolsep}{0in}

% Sections formatting
\titleformat{\section}{
  \vspace{-4pt}\scshape\raggedright\large
}{}{0em}{}[\color{black}\titlerule \vspace{-5pt}]

% Ensure that generate pdf is machine readable/ATS parsable
\pdfgentounicode=1

%-------------------------
% Custom commands
\newcommand{\resumeItem}[1]{
  \item\small{
    {#1 \vspace{-2pt}}
  }
}

\newcommand{\resumeSubheading}[4]{
  \vspace{-2pt}\item
    \begin{tabular*}{0.97\textwidth}[t]{l@{\extracolsep{\fill}}r}
      \textbf{#1} & #2 \\
      \textit{\small#3} & \textit{\small #4} \\
    \end{tabular*}\vspace{-7pt}
}

\newcommand{\resumeSubSubheading}[2]{
    \item
    \begin{tabular*}{0.97\textwidth}{l@{\extracolsep{\fill}}r}
      \textit{\small#1} & \textit{\small #2} \\
    \end{tabular*}\vspace{-7pt}
}

\newcommand{\resumeProjectHeading}[2]{
    \item
    \begin{tabular*}{0.97\textwidth}{l@{\extracolsep{\fill}}r}
      \small#1 & #2 \\
    \end{tabular*}\vspace{-7pt}
}

\newcommand{\resumeSubItem}[1]{\resumeItem{#1}\vspace{-4pt}}

\renewcommand\labelitemii{$\vcenter{\hbox{\tiny$\bullet$}}$}

\newcommand{\resumeSubHeadingListStart}{\begin{itemize}[leftmargin=0.15in, label={}]}
\newcommand{\resumeSubHeadingListEnd}{\end{itemize}}
\newcommand{\resumeItemListStart}{\begin{itemize}}
\newcommand{\resumeItemListEnd}{\end{itemize}\vspace{-5pt}}

%-------------------------------------------
%%%%%%  RESUME STARTS HERE  %%%%%%%%%%%%%%%%%%%%%%%%%%%%


\begin{document}

%----------HEADING----------

\begin{center}
    \textbf{\Huge \scshape Semen Tenishchev} \\ \vspace{1pt}
    \small +7 962 705 41 49 $|$ \href{mailto:tenischev.semen@gmail.com}{\underline{tenischev.semen@gmail.com}} $|$
    \href{https://github.com/Tenischev}{\underline{github.com/Tenischev}} $|$
    \href{https://profile.codersrank.io/user/Tenischev}{\underline{codersrank.io/Tenischev}}
\end{center}


%-----------EDUCATION-----------
\section{Education}
  \resumeSubHeadingListStart
    \resumeSubheading
      {ITMO University}{Saint Petersburg, Russia}
      {Phd of Mathematical modeling, numerical methods and program complexes}{2018 -- Present}
    \resumeSubheading
      {ITMO University}{Saint Petersburg, Russia}
      {Master of Applied Mathematics and Computer Science}{2016 -- 2018}
    \resumeSubheading
      {ITMO University}{Saint Petersburg, Russia}
      {Bachelor of Applied Mathematics and Computer Science}{2012 -- 2016}
  \resumeSubHeadingListEnd


%-----------EXPERIENCE-----------
\section{Experience}
  \resumeSubHeadingListStart

    \resumeSubheading
      {Product Owner}{February 2020 -- Present}
      {Deutsche Telekom IT RUS (former T-Systems RUS)}{Saint Petersburg, Russia}
      \resumeItemListStart
        \resumeItem{Agile team of 7 people, company leads by SAFe framework}
        \resumeItem{Discussions with a customers, clarification of requirements, creation and prioritization of stories for the team}
        \resumeItem{Two new microservices were created and run into production}
        \resumeItem{Spring Boot, Kafka, Flyway, PostgreSQL and Docker were used in development, REST API was defined by OpenAPI standard}
      \resumeItemListEnd

    \resumeSubheading
      {Developers team lead}{March 2019 -- February 2020}
      {T-Systems RUS}{Saint Petersburg, Russia}
      \resumeItemListStart
        \resumeItem{Communicate with a product owner and analysts}
        \resumeItem{Defining architectural and technical solutions}
        \resumeItem{Splitting stories to a tasks for a team}
        \resumeItem{Conducting F2F with a team members, support in the development}
    \resumeItemListEnd

    \resumeSubheading
      {Senior Java developer}{December 2018 -- February 2019}
      {T-Systems RUS}{Saint Petersburg, Russia}
      \resumeItemListStart
      \resumeItem{Develop and support of legacy full-stack application written on JavaEE with Struts as the frontend and Oracle-DB}
      \resumeItem{Develop and support of legacy Windows thick-client written on JavaScript and C\#}
    \resumeItemListEnd

    \resumeSubheading
      {Java developer}{October 2017 -- November 2018}
      {T-Systems RUS}{Saint Petersburg, Russia}
    \resumeSubheading
      {Junior Java developer}{February 2016 -- September 2017}
      {T-Systems RUS}{Saint Petersburg, Russia}

  \resumeSubHeadingListEnd


%-----------PROJECTS-----------
\section{Free time projects}
    \resumeSubHeadingListStart
      \resumeProjectHeading
          {\textbf{AsyncAPI Java generator} $|$ \emph{Java, Spring Boot, JavaScript}}{April 2020 -- Present}
          \resumeItemListStart
            \resumeItem{Developed a module for a AsyncAPI code generator to create Spring Boot based code template like it does in OpenAPI}
            \resumeItem{Several async protocols were supported: kafka, mqtt, amqp}
            \resumeItem{Also several adjustments in central generator code}
          \resumeItemListEnd
      \resumeProjectHeading
          {\textbf{Traffic light} $|$ \emph{Java, Gradle, Docker}}{July 2019 -- August 2019}
          \resumeItemListStart
            \resumeItem{Developed a three microservices, first with DB and CRUD REST API to store "news", second is frontend app to display the "news" and status of build from Jenkins as a Traffic light, third is frontend app to create "news"}
            \resumeItem{All services has a build pipeline to create Docker images}
          \resumeItemListEnd
    \resumeSubHeadingListEnd



%
%-----------PROGRAMMING SKILLS-----------
\section{Technical Skills}
 \begin{itemize}[leftmargin=0.15in, label={}]
    \small{\item{
     \textbf{Languages}{: Java SE and EE, SQL (PL/SQL, Postgres), JavaScript, HTML/CSS, C++, C} \\
     \textbf{Frameworks}{: Spring Boot, Apache Camel, Flyway, JUnit, OpenAPI, React} \\
     \textbf{Tools}{: Jira, GitLab, IntelliJ, Docker, Rancher, Kafka, IMB MQ}
    }}
 \end{itemize}


%-------------------------------------------
\end{document}
